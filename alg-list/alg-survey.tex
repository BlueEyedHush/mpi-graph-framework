\documentclass[preprint]{sig-alternate-05-2015}

\include{mac-includes}

\usepackage{makecell}

\begin{document}

\title{Spectral Made Practical}

\maketitle

\begin{abstract}
%
%% Motivation/problem statement: Why do we care about the problem? What
%% practical, scientific, theoretical or artistic gap is your research
%% filling?
Abstract. Abstract.  Abstract. Abstract.  Abstract. Abstract.  Abstract.
Abstract.  Abstract. Abstract.  Abstract. Abstract.  Abstract. Abstract.
Abstract. Abstract.  Abstract. Abstract.  Abstract. Abstract.  Abstract.
Abstract.  Abstract. Abstract.  Abstract. Abstract.  Abstract. Abstract.
Abstract. Abstract.  Abstract. Abstract.  Abstract. Abstract.  Abstract.
Abstract.  Abstract. Abstract.  Abstract. Abstract.  Abstract. Abstract.
Abstract. Abstract.  Abstract. Abstract.  Abstract. Abstract.  Abstract.
Abstract.  Abstract. Abstract.  Abstract. Abstract. 
%% Methods/procedure/approach: What did you actually do to get your
%% results? (e.g. analyzed 3 novels, completed a series of 5 oil
%% paintings, interviewed 17 students)

%% Results/findings/product: As a result of completing the above
%% procedure, what did you learn/invent/create?

%% Conclusion/implications: What are the larger implications of your
%% findings, especially for the problem/gap identified in step 1?
%The Slim Fly topology uses fewer routers and has a 25\% lower construction
%cost than a Dragonfly network with a comparable number of endpoints. 
%
\end{abstract}

% \vspace{1em}
% 
% \macb{Code:} \maciej{TODO: provide a link}
% 
% \vspace{1em}


\section{Introduction}

\section{Graph Problems and Algorithms that Matter}

(X) Answering simple topological queries over a given graph

-- give me the degree of a given vertex

-- give me the neighbors of this vertex

-- give me the neighbors of the neighbors...

-- check if these two vertices are connected

-- list all the edges

-- list all the vertices


(X) Computing Centralities

-- PageRank

-- Betweenness Centrality

-- Stress Centrality

-- Closeness Centrality

-- Degree Centrality

-- Katz centrality

-- Cross-clique Centrality

-- K-path Centrality

-- Eigenvector Centrality

-- Graph Centralization


(X) Traversals

-- BFS order

-- DFS order

-- BFS distances

-- BFS predecessor-successor tree

-- Cuthill-McKee order


(X) Counting Triangles

-- Count the total of triangles TCT

-- Count triangles that each vertex is in, TCV. Motivated: this enables computing a vertex clustering coefficient.

-- triangle enumeration TE: list all the triangles.

-- deciding whether G contains a triangle

-- for each triangle in G, list the participating nodes


(X) Shortest Path Finding

-- Single Source

-- All pairs


(X) Graph Isomorphism problems


(X) Graph Realization problems


(X) Computing Spanning Trees


(X) Computing Minimum Spanning Trees


(X) Deriving Minimum Graph Colorings


(X) Network Reduction (Extract the backbone structure of undirected weighted network).


-- K-core decompositon

-- MST

-- Global Weight Threshold


(X) Flow Problems

-- Maximum Flow

-- Circulation Problems


(X) Connectivity


(X) Hamiltonicity problems


(X) Clique Problems

-- Max Clique Finding

-- Finding all k-cliques


(X) Matchings


(X) Independent Sets


(X) Node Covers


(X) Edge Covers


(X) Topological Sorting


(X) Transitivity Problems


-- Transitivity Reduction


(X) Cycle detection


(X) Cuts


(X) Clustering


(X) graph pattern matching


(X) VARIOUS Graph Properties:

Survey:~\cite{hu2013survey}.  Some example list:

\begin{itemize}
%
\item Graph size: $n$, $m$.
%
\item Degree distribution~\cite{hu2013survey},
%
\item Average degree~\cite{hu2013survey},
%
\item Power Law exponent~\cite{hu2013survey},
%
\item Graph density~\cite{hu2013survey},
%
\item t-step path matrix~\cite{hu2013survey},
%
\item Shortest Path matrix~\cite{hu2013survey},
%
\item Average shortest path length~\cite{hu2013survey},
%
\item Closeness centrality (vertex property)~\cite{hu2013survey},
%
\item Radius (vertex property)~\cite{hu2013survey},
%
\item Diameter~\cite{hu2013survey},
%
\item BC(v)~\cite{hu2013survey},
%
\item Assortativity~\cite{hu2013survey},
%
\item Clustering coefficient (v)~\cite{hu2013survey},
%
\item Average clustering coefficient~\cite{hu2013survey},
%
\item GLobal clustering coefficient, transitivity~\cite{hu2013survey},
%
\item Self-similarity~\cite{hu2013survey},
%
\item Spectrum / Eigenvalue distribution~\cite{hu2013survey},
%
\item Cut(S), Ratio cut(S), Normalized cut(S), Weighted cut~\cite{hu2013survey},
%
\item Association(S), Ratio assoc(S), normalized assoc(S), Weighted assoc(S)~\cite{hu2013survey},
%
\item Conductance(S), Expantion(S)~\cite{hu2013survey},
%
\item Quadratic form (more general than assoc and cut)~\cite{hu2013survey},
%
\item Modularity(S)~\cite{hu2013survey},
%
\item Cohesion(S) - generalizes notion of cuts (triangles across cuts)~\cite{hu2013survey},
%
\item Triangle count, TC(v), Triangle cut(S), Triangle assoc(S)~\cite{hu2013survey},
%
\item Largest (first?) eigenvalue $\lambda_1$~\cite{wu2010reconstruction} of the adj
matrix $A$.
%
\item Second largest eigenvalue $\nu_2$~\cite{wu2010reconstruction} of the normal
matrix $N = D^{-1} A$.
%
\item $Q$: modularity indicates the goodness of the community
structure~\cite{wu2010reconstruction}.
%
\item $C$: transitivity measure is one type of clustering coefficient measure;
characterizes the presence of local loops in G~\cite{wu2010reconstruction}.
%
\end{itemize}


\bibliographystyle{abbrv}
\bibliography{alg-survey}

\end{document}

